\section{Tasks}

Task is a set of actions which should be performed on the node.
The task must also contain the list of the system requirements which is necessary for its execution.

Each task accepted by the system has a unique identifier (UUID)\cite{uuid}.
It's possible to track, cancel or gather the results of the task using this identifier.

\newpage

After the task is accomplished the following metrics are collected for future proceeding:

\begin{itemize}
    \item CPU time spent
    \item GPU time spent
    \item Incomming network traffic
    \item Outbound network traffic
    \item Total elapsed time
\end{itemize}

These metrics are used for reward calculation, QA and so on.

\subsection{Task specification}

To describe the task the following type specification should be used:

\begin{minted}{erlang}
-type xpu() :: <<"cpu">> | <<"nvidia">> | <<"amd">>.

-type step() :: #{
    <<"name">>    => binary(),
    step_module() := step_args()
}.

-type task() :: #{
    <<"name">>                   => binary(),
    <<"xpu">>                    := xpu(),
    <<"xpu-cpu-fallback">>       := boolean(),
    <<"xpu-n">>                  := pos_integer(),
    <<"xpu-fallback-wait-time">> => pos_integer(),
    <<"pending-ttl">>            => pos_integer(),
    <<"ttl">>                    => pos_integer(),
    <<"spaces">> := #{
        xpu() := binary()
    },
    <<"steps">> := [step()]
}.
\end{minted}

\begin{itemize}
    \item name - human readable description of the task
    \item xpu - the type of execution unit, can be \textbf{cpu}, \textbf{nvidia} or \textbf{amd} if the task requires GPU
    \item xpu-cpu-fallback - in case if xpu is set to use GPU but no node with GPU found during \textbf{xpu-fallback-wait-time} time task will be assigned to the node with CPU only
    \item xpu-n - minimum amount of the required execution units eg. amount of GPUs or CPU cores
    \item pending-ttl - task will be canceled if no node is found during \textbf{pending-ttl} period
    \item ttl - maximum task execution time
    \item spaces - the list of specific isolated environments per xpu type
    \item steps - the list of actions which should be performed on the node
\end{itemize}

More detailed specification may be found in the project OSS\cite{oss} documents.

\subsection{Task examples}

Task can be described using both JSON or YAML formats.

GPU mining task example described in YAML format:

\inputminted{yaml}{claymore-cuda-mining-task.yaml}

\newpage
Neural Style Transfer task example described in JSON:

\inputminted{json}{neural-style-task.json}
